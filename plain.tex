%! TeX Program = LuaLaTeX

%% ==> preamble
\documentclass{article}
\pagestyle{empty}

\makeatletter
\def\ltj@stdmcfont{SourceHanSerifSC}
\def\ltj@stdyokojfm{eva/{nstd,smpl}}
\makeatother

\usepackage{luatexja}

\usepackage{geometry}
\geometry{paper=a4paper,
          includemp,
          hcentering,
          hmargin=2.4cm,
          tmargin=1.8cm,
          bmargin=1.6cm,
          marginparsep=0.6cm,
          marginparwidth=1.4cm,
          verbose}
\marginparpush=1mm

\usepackage{unicode-math}
\setmathfont{Libertinus Math}

\usepackage{fontspec}
\newfontfeature{microtype}{protrusion=default;expansion=deafult}
\defaultfontfeatures{microtype}
\setmainfont{Linux Libertine O}
\setsansfont{Linux Biolinum O}
\adjustspacing=2
\protrudechars=2

\usepackage[normalem]{ulem}

\newcounter{wordexplcnt}
\newcommand\wordexpl[2]{%
           \stepcounter{wordexplcnt}%
           {#1}\nolinebreak[4]\textsuperscript{\tiny\thewordexplcnt}%
           \marginpar{\hbox to 1.4cm {\llap{\textsuperscript{\tiny\thewordexplcnt}}\nolinebreak[4]\mcfamily\small{#2}\hss}}}
\newcounter{wordreplcnt}
\newcommand\wordrepl[2]{%
           \stepcounter{wordreplcnt}%
           \uwave{#2}\nolinebreak[4]\textsuperscript{\tiny{}*\thewordreplcnt}%
           \marginpar{\hbox to 1.4cm {\llap{\textsuperscript{\tiny{}*\thewordreplcnt}}\nolinebreak[4]\ \nolinebreak[4]\mcfamily\small\textit{#1}\hss}}}
\newif\ifFiNaL \FiNaLtrue %<>Toggle
\newcommand\maybeblock[2]{%
           \ifFiNaL{\uline{#1#2}}%
           \else{\hbox{#1\uline{\phantom{#2#2}}}}\fi}
\def\;{\hbox to .5\zw{;\hss}}
\def\[{\hbox to .5\zw{\hss[}}
\def\]{\hbox to .5\zw{]\hss}}
\def\CR{\\{}}
%% preamble <==

\begin{document}

\section*{\sffamily {\TeX} and Her Friend: Hyphenation}
It's \maybeblock{b}{etter} to break a word with a \wordexpl{hyphen}{断词用连字符} than to stretch \wordexpl{interword}{单词间的{\[}术语{\]}} spaces too much.
Therefore {\TeX} tries to divide words into \wordexpl{syllables}{音节} when there's no good \wordexpl{alternative}{可供替代的} \maybeblock{a}{vailable}.

But computers are \wordexpl{notoriously}{众所周知的{\;}臭名昭著的} bad at hyphenation. 
When the \wordexpl{typesetting}{排版{\;}组版} of newspapers began to be \maybeblock{f}{ully} automated, jokes about ``th\hbox{e-r}apists who pr\hbox{e-a}ched on we\hbox{e-k}nights'' soon began to \wordrepl{circulate}{spread}.

It's not hard to understand why machines have behaved poorly at this \maybeblock{t}{ask}, because hyphenation is quite a difficult problem.
For example, the word `record' is \maybeblock{s}{upposed} to be broken as `rec-ord' when it is a noun, but `re-cord' when it is a verb.
The word `hyphenation' itself is somewhat \wordrepl{exceptional}{extraordinary};
if `hy-phen-a-tion' is compared to \maybeblock{s}{imilar} words like `con-cat-e-na-tion', it's not immediately clear why the `n' should be \maybeblock{a}{ttached} to the `e' in one case but not the other.
Examples like `\hbox{dem-on-stra-tion}' vs\@. `\hbox{de-mon-stra-tive}' show that the \wordrepl{alteration}{change} of two letters can actually affect hyphens that are nine \maybeblock{p}{ositions} away.

A good solution to the problem was discovered by Frank M. Liang during 1980-1982, and {\TeX} \wordrepl{incorporates}{includes} the new method.
Liang's \wordexpl{algorithm}{算法{\[}计{\]}} works quickly and finds nearly all of the \wordrepl{legitimate}{reasonable} places to insert hyphens;
yet it makes few if any errors, and it takes up \wordexpl{comparatively}{相对地{\;}相较而言地} little space in the computer.
Moreover, the method is \maybeblock{f}{lexible} enough to be adapted to any language, and it can also be used to hyphenate words in two languages \wordrepl{simultaneously}{at the same time}.
Liang's Ph.D. \wordexpl{thesis}{论文}, published by Stanford University's Department of Computer Science in 1983, explains how to take a dictionary of hyphenated words and \maybeblock{t}{each} it to {\TeX};
i.e., it explains how to compute tables by which {\TeX} will be able to reconstruct most of the hyphens in the given dictionary, without error.

\smallskip
\begin{flushright}
      \sffamily\itshape\small\parskip=1mm
      If all problems of hyphenation have not been solved,\CR
      at least some progress has been made since that night,\CR
      when according to legend, an RCA Marketing Manager received\CR
      a phone call from a disturbed customer. His 301 had just hyphenated ``God''.\par
      --- \emph{PAUL E. JUSTUS}, There's More to Typesetting Than Setting Type \emph{(1972)}\par
      \smallskip
      The committee skeptically re-\CR
      commended more study for a bill\CR
      to require warning labels on rec-\CR
      ords with subliminal messages re-\CR
      corded backward.\par
      --- \emph{THE PENINSULA TIMES TRIBUNE} \emph{(April 28, 1982)}
\end{flushright}

\section*{\sffamily Appendix: The Name of the Game}
English words like `technology' \wordrepl{stem}{originate} from a Greek root beginning with the letters $τεχ$\ldots;
and this same Greek word means \textit{art} as well as technology.
Hence the name {\TeX}, which is an uppercase \maybeblock{f}{orm} of $τεχ$.

\wordexpl{Insiders}{知情之人} pronounce the $χ$\, of {\TeX} as a Greek chi, not as an `x', so that {\TeX} rhymes with the word \textit{blecchhh}.
It's the `ch' sound in Scottish words like \textit{loch} or German words like \textit{ach};
it's a Spanish `j' and a Russian `kh'.
When you say it \maybeblock{c}{orrectly} to your computer, the terminal may become \wordexpl{slightly}{略微{\;}稍微} \wordexpl{moist}{潮湿而微雾}.

The purpose of this pronunciation \maybeblock{e}{xercise} is to remind you that {\TeX} is primarily concerned with high-quality technical \wordexpl{manuscripts}{原稿{\;}手稿}:
Its emphasis is on art and technology, as in the underlying Greek word.
If you merely want to produce a \wordrepl{passably}{satisfactorily} good document\,--\,something acceptable and basically readable but not really beautiful\,--\,a simpler system will usually \wordrepl{suffice}{do}.
With {\TeX} the goal is to produce the finest quality;
this requires more attention to detail, but you will not find it much harder to go the extra distance, and you'll be able to take special \maybeblock{p}{ride} in the finished product.

\smallskip
\begin{flushright}
      \sffamily\itshape\small\parskip=1mm
      They do certainly give\CR
      very strange and new-frangled names to diseases.\par
      --- \emph{PLATO}, The Republic, \emph{Book 3 (c\@. 375 B.C.)}\par
      \smallskip
      Technique! The very word is like the shriek\CR
      Of outraged Art. It is the idiot name\CR
      Given to effort by those who are too weak,\CR
      Too weary, or too dull to play the game.\par
      --- \emph{LEONARD BACON}, Sophia Trenton \emph{(1920)}
\end{flushright}

\medskip\vtop to 0pt{\hbox{\kern-2.4cm\hbox to \pagewidth{\hss\scriptsize\itshape{Orig text by Donald E. Knuth. Edited (simplified) and typeset by Jing Huang. Created by Lua\kern-.12em{\TeX} with micro typographic techniques. Dated \today. Revision no: 2.\hss}}}}

\end{document}
